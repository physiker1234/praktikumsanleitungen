\documentclass{tudphygp}
\usepackage{xcolor,verbatim,moreverb,listings}
\lstset{basicstyle=\footnotesize\ttfamily}

\author{Jeffrey Kelling,Felix Lemke,Stefan Majewsky}
\versuch[]{Styleguide}{Styleguide}

% Darstellung von Codebeispielen: Zuerst muss der Code zwischen \begin{code} und \end{code} notiert werden. Danach kann das Codebeispiel mit \printcode ausgegeben werden. (Achtung: Nach \end{code} wird aus technischen Gr�nden alles weitere auf derselben Zeile ignoriert!)
\def\code{\begingroup\verbatimwrite{\jobname.code.tmp}}
\def\endcode{\endverbatimwrite\endgroup}
\newcommand\printcode{
 \par\fcolorbox{gray!20}{gray!20}{
  \begin{minipage}{\linewidth*3/5-1.5em}
   \vspace*{-0.5em}
   \lstinputlisting{\jobname.code.tmp}
   \vspace*{-0.5em}
  \end{minipage}
 }\hspace{0.4em}\fbox{
  \begin{minipage}{\linewidth*2/5-1.5em}
   \input{\jobname.code.tmp}
  \end{minipage}
 }\par
}

\def\TODO#1{\par{\color{red}\textbf{TODO:~}#1}\par}

\begin{document}

\maketitle

\section{Vorwort}

Das vorliegende Dokument fasst die Regeln zusammen, die wir f�r uns w�hrend der �berarbeitung der Grundpraktikums-Anleitungen erarbeitet haben, um einen hohen Grad an Konsistenz zwischen den verschiedenen Anleitungen zu erreichen, und grundlegende typografische Regeln einfach verst�ndlich zu dokumentieren. Wir hoffen, dass diese Ratschlagssammlung den uns nachfolgenden Bearbeitern erm�glicht, den Satzstil der Anleitungen aufzugreifen und �berarbeitungen in einer �bersichtlichen und angenehm lesbaren Form einzubringen.

\section{Dokumentstruktur}

Die Anleitungen sollten grundlegend wie folgt gegliedert sein:
\begin{itemize}
 \item Aufgabenstellung
 \item Theoretische Grundlagen
 \item Versuchsdurchf�hrung
 \item (nach Bedarf) Anhang
 \item (implizit) Fragen/Literatur
\end{itemize}
Wir empfehlen auch die Aufteilung der Kapitel entlang dieser Struktur, allerdings kann man davon auch abweichen. Die Aufgabenstellung sollte in Platzanleitung und Anleitung identisch sein, und den grundlegenden Leitfaden zum Aufbau der folgenden Abschnitte darstellen.

Sollten sich Aufgabenstellungen nur auf Physiker beziehen wird folgende Formatierung empfohlen:
\begin{code}
 \begin{itemize}
  \item \emph{(F�r Physikstudenten)} $\ldots$
 \end{itemize}
\end{code}
\printcode

Am Beispiel des Versuches "`Innere Reibung von Fl�ssigkeiten"' :
\begin{itemize}
 \item Geben Sie die Koeffizienten $A$ und $b$ der \emph{Andradeschen Gleichung}
 \item \emph{(F�r Physikstudenten)} Geben Sie die Standardabweichungen der Andradeschen Koeffizienten an. Berechnen Sie die \emph{Reynoldssche Zahl} $Re$ f�r die h�chste Temperatur.
\end{itemize}


\section{Textsatz}

Bei Definitionen und Aufz�hlungen derselben soll die \texttt{description}-Umgebung mit abschlie�endem Doppelpunkt in den Beschriftungen verwendet werden. Dasselbe gilt f�r sonstige betitelte Aufz�hlungen, hier sollte jedoch auf den Doppelpunkt verzichtet werden.
%\TODO{Codebeispiel?} -- Nein.

F�r Zitate werden deutsche Anf�hrungszeichen verwendet.

\begin{code}
 Falsch: ``englische Variante'' \\
 Richtig: "`deutsche Variante"'
\end{code}
\printcode

\section{Mathematischer Formelsatz}

Beim Einbinden von Formeln in den Mengentext sollte mit Augenma� vorgegangen werden:
\begin{itemize}
 \item Zu hohe Formeln erh�hen die Zeilenabst�nde und verringern den Grauwert, wodurch der Mengentext unruhig wirkt und schlechter zu lesen ist.
 \item Bei zu langen Formeln besteht die Gefahr, dass sie nicht oder nicht richtig umgebrochen werden, was das Lesen erschwert.
\end{itemize}

Mehrzeilige herausgestellte Formeln sollten ausschlie�lich mit der \texttt{align}-Umgebung gesetzt werden. Wir empfehlen, die Ausrichtung auch im Code deutlich zu machen, damit andere Bearbeiter sich schneller einlesen k�nnen:

\begin{code}
 Formeln umstellen ist nicht schwer:
 \begin{align*}
  a^2 + b^2 &= c^2 \\
        b^2 &= c^2 - a^2
 \end{align*}
\end{code}
\printcode

Formelnummern sollten nur vergeben werden, wenn die Formel wirklich sp�ter referenziert werden soll, oder einen f�r das Praktikum wichtigen Sinninhalt transportiert:

\begin{code}
 Formeln umstellen ist nicht schwer:
 \begin{align}
  a^2 + b^2 &= c^2 \nonumber\\
        b^2 &= c^2 - a^2
 \end{align}
\end{code}
\printcode

Ein Name bzw. Beschreibung kann einer Formelzeile in einem \texttt{align} hinter einem zweiten \texttt{\&} hinzugef�gt werden.

\begin{code}
 Formeln umstellen ist nicht schwer:
 \begin{align*}
  a^2 + b^2 &= c^2 &\text{Pythagoras}\\
        b^2 &= c^2 - a^2
 \end{align*}
\end{code}
\printcode


Bei vektoriellen Gr��en ist darauf zu achten, dass ein eventueller Anstrich (wie in: $x'$) sich nicht mit dem Vektorpfeil �berschneidet. Daf�r ist bei Gr��en ohne unteren Index ein kleiner Abstand bzw. bei Gr��en mit Index ein kleiner negativer Abstand zwischen dem unteren Index und dem Anstrich zu verwenden:

\begin{code}
 Falsch: $\vec x'$, $\vec a_f'$
 
 Richtig: $\vec x\,'$, $\vec a_f\!'$
\end{code}
\printcode

Bei der Bildung von Gleichungsketten mit Trennsymbolen wie "`$\Rightarrow$"' oder W�rtern wie "`mit"' und "`und"' ist in herausgestellten Formeln um die Trennw�rter bzw. -symbole ein Geviertabstand einzuf�gen. Steht die Formel im Text, so sollten im Allgemeinen keine Trennsymbole verwendet werden, sondern die Relation textuell formuliert werden, wobei die Verwendung von \lstinline{\text} vermieden werden sollte.

\begin{code}
 \[
  a = b \quad\text{und}\quad c = d
 \]
 Daraus folgt $a = d$ wegen $b = c$.
\end{code}
\printcode

Am Beginn einer Zeile (insb. in mehrzeiligen Formeln) sollte der Geviertabstand nat�rlich entfallen:

\begin{code}
 Formeln umstellen ist nicht schwer:
 \begin{align}
       a^2 + b^2 &= c^2 \nonumber\\
  \to\quad b^2 &= c^2 - a^2
 \end{align}
\end{code}
\printcode

\TODO{Platzanleitung: �bernahme der Formelnummern mit refequation}
\TODO{Formulierung der Aufgabenstellungen immer mit vollst�ndigen S�tzen}
\TODO{Zahlen und Gr��en immer mit num und SI, common-sense �ber das Aufschreiben von Einheiten}
\TODO{Fettdruck nur bei Sicherheitshinweisen, ansonsten Betonungen bzw. Markierung neuer Begriffe mit emph}
\TODO{Schreibung von Abk�rzungen: "`z.~B."', aber auch "`s.~Tabelle~{\textbackslash}ref"'}
\TODO{Verwendung von Trennlinien in Tabellen}
\TODO{Jeffrey: Anpassen der align-Regeln (Beschriftung rechts mit zwei Et-Zeichen abtrennen)}

\end{document}
