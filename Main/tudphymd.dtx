% \iffalse meta-comment
%
% Copyright 2009 by Felix Lemke <lemke.felix@ages-skripte.org>
% Copyright 2009 by Stefan Majewsky <majewsky.stefan@ages-skripte.org>
% -----------------------------------
%
% This file may be distributed and/or modified under the
% conditions of the LaTeX Project Public License, either version 1.2
% of this license or (at your option) any later version.
% The latest version of this license is in:
%
%     http://www.latex-project.org/lppl.txt
%
% and version 1.2 or later is part of all distributions of LaTeX
% version 1999/12/01 or later.
%
% \fi
%
% \iffalse
%<style>\NeedsTeXFormat{LaTeX2e}[2005/12/01]
%<style>\ProvidesPackage{tudphymd}
%<style>   [2009/04/07 0.9beta2 Paket mit speziellen Vereinfachungen f�r die Praktikumsanleitungen]
%
%<*driver>
\documentclass{ltxdoc}
 \usepackage[ngerman]{babel}
 \usepackage[latin1]{inputenc}
 \usepackage[T1]{fontenc}
 \usepackage{tudphymd}
 \usepackage{calc}
 \usepackage{booktabs,tabularx}
 \usepackage{amsmath}

 \parindent0pt
 \parskip.5em
 \EnableCrossrefs
 \CodelineIndex
 \RecordChanges
 \begin{document}
  \DocInput{tudphymd.dtx}
 \end{document}
%</driver>
%
% \fi
%
% \CheckSum{178}
%
% \CharacterTable
%  {Upper-case    \A\B\C\D\E\F\G\H\I\J\K\L\M\N\O\P\Q\R\S\T\U\V\W\X\Y\Z
%   Lower-case    \a\b\c\d\e\f\g\h\i\j\k\l\m\n\o\p\q\r\s\t\u\v\w\x\y\z
%   Digits        \0\1\2\3\4\5\6\7\8\9
%   Exclamation   \!     Double quote  \"     Hash (number) \#
%   Dollar        \$     Percent       \%     Ampersand     \&
%   Acute accent  \'     Left paren    \(     Right paren   \)
%   Asterisk      \*     Plus          \+     Comma         \,
%   Minus         \-     Point         \.     Solidus       \/
%   Colon         \:     Semicolon     \;     Less than     \<
%   Equals        \=     Greater than  \>     Question mark \?
%   Commercial at \@     Left bracket  \[     Backslash     \\
%   Right bracket \]     Circumflex    \^     Underscore    \_
%   Grave accent  \`     Left brace    \{     Vertical bar  \|
%   Right brace   \}     Tilde         \~}
%
% \DoNotIndex{\ifthenelse,\@ifundefined,\equal,\fi,\ifpdf,\NOT,\OR,\else,\IfFileExists,\@for,\do}
% \DoNotIndex{\\,\arabic,\bfseries,\centering,\hfill,\hline,\item,\Large,\large,\linewidth,\makebox,\newpage,\parbox,\selectfont,\setlength,\small,\textsc,\texttt,\thepage,\today,\vskip,\vspace,\flushleft,\LARGE,\section,\flushright,\newline,\parskip,\resizebox,\vfill}
% \DoNotIndex{\begin,\end,\csname,\endcsname,\def,\edef,\endinput,\expandafter,\g@addto@macro,\global,\let,\newcommand,\newcounter,\PassOptionsToClass,\ProcessOptions,\protect,\relax,\renewcommand,\RequirePackage,\setcounter,\stepcounter,\includegraphics,\noexpand,\ensuremath,\input}
% \DoNotIndex{\@temp@@newList,\@temp@Item}
% \DoNotIndex{\{,\},\frac,\left,\right,\mathrm,\partial}
%
% \changes{v0.1}{2009/04/07}{Initial Version}
% \changes{v0.1.1}{2009/08/04}{New commands errSI and errnum}
% \changes{v0.2}{2009/08/10}{sublabel added}
%
% \GetFileInfo{tudphymd.sty}
%
% \title{Das \textsf{tudphymd}-Paket\thanks{Dieses Dokument entspricht \textsf{\filename} Version \fileversion, \filedate}}
% \author{Felix Lemke, Stefan Majewsky \\ \texttt{info@ages-skripte.org}}
%
% \maketitle
% \begin{abstract}
%  Dieses Paket stellt f�r mit der \textsf{tudphygp}-Dokumentklasse gesetzten Dokumente einige Zusatzfunktionen bereit.
% \end{abstract}
%  \newpage
% \tableofcontents
% \newpage
%
% \section{Mathematische Befehle}
% In diesem Abschnitt werden Befehle definiert, die das Schreiben von Formeln einfacher und lesbarer machen.
%
% \subsection{Differential- und Integralrechnung}
% \begin{tabularx}{\linewidth}{lXcc}
%  \textbf{Befehl} & \textbf{Beschreibung}                                  & \textbf{Beispielcode} & \textbf{Ausgabe}\\\toprule
%  |\dquot|        & Differentialquotient (Ordnung als optionales Argument) & |$\dquot[2]xt$| & $\dquot[2]xt$ \\
%  |\pdquot|       & partieller Differentialquotient                        & |$\pdquot px$| & $\pdquot px$ \\
%  |\diff|         & Differentialoperator                                   & |$W \cdot \diff x$| & $W \cdot \diff x$ \\
%  |\dif|          & Differentialoperator mit Abstand (f�r Integrale)       & |$\int f\dif x$| & $\int f\dif x$ \\
%  |\Int|          & Integralsymbol f�r herausgestellte Formeln             & |$\Int_a^b$| & $\Int_a^b$
% \end{tabularx}
%
% \subsection{Klammern}
% Damit man nicht st�ndig |\left| und |\right| anwenden muss, wurden ein paar typische Klammerpaare vordefiniert.
%
% \begin{tabularx}{\linewidth}{lXcc}
%  \textbf{Befehl} & \textbf{Beschreibung} & \textbf{Beispielcode} & \textbf{Ausgabe}\\\toprule
%  |\abs| & Betragsstriche & |$\abs{-\frac12}$| & $\abs{-\frac12}$ \\[1ex]
%  |\brk| & eckige Klammern & |$\brk{\int\diff x}$| & $\brk{\int\diff x}$ \\[1ex]
%  |\brc| & geschweifte Klammern & |$\brc{\sum x^2}$| & $\brc{\sum x^2}$ \\[1ex]
%  |\kla| & runde Klammern & |$\kla{b^2 + c}$| & $\kla{b^2 + c}$
% \end{tabularx}
%
% \subsection{Angabe fehlerbehafteter Gr��en}
% In der |tudphygp|-Dokumentklasse ist schon das Paket |sistyle| eingebunden, dass den einfachen Satz von Werten und Einheiten �ber die Befehle |\SI| und |\num| erlaubt. Die Befehle |\errSI| und |\errnum| erweitern dies auf fehlerbehaftete Gr��en. Die folgende Tabelle zeigt, wie und wann diese Befehle angewendet werden.
%
% \begin{tabularx}{\linewidth}{Xcc}
%  \textbf{Anwendungsfall} & \textbf{Beispielcode}      & \textbf{Ausgabe}         \\\toprule
%   nur Wert               & |\errnum{10}{3}|           & \errnum{10}{3}           \\
%   mit Zehnerpotenz       & |\errnum[e5]{10}{3}|       & \errnum[e5]{10}{3}       \\
%   mit Einheit            & |\errSI{10}{3}{A.m^2}|     & \errSI{10}{3}{A.m^2}     \\
%   Einheit und Zehnerp.   & |\errSI[e5]{10}{3}{A.m^2}| & \errSI[e5]{10}{3}{A.m^2} \\
% \end{tabularx}
%
% \subsection{Diverse}
%
% \DescribeMacro{\eu}
% Die Eulersche Zahl sollte immer in Serifenschrift gesetzt werden. Da man das aber viel zu oft vergisst und zum Beispiel |$e^x$| schreibt, gibt es den Befehl |\eu|, der die Eulersche Zahl erzeugt. Das erste und einzige Argument wird zur Potenz.
%
% \begin{center}\begin{tabular}{ccc}
%          & \textbf{Beispielcode}       & \textbf{Ausgabe}          \\\toprule
%  falsch  & |$f(x,y) = e^{x^2 + y^2}$|  & $f(x,y) = e^{x^2 + y^2}$  \\
%  richtig & |$f(x,y) = \eu{x^2 + y^2}$| & $f(x,y) = \eu{x^2 + y^2}$ \\
% \end{tabular}\end{center}
%
% \section{Mathematischer Formelsatz}
%
% \DescribeEnv{refequation}
% Damit man sich zum Beispiel in einer Platzanleitung auf eine Gleichung aus der Hauptanleitung beziehen kann, gibt es die |refequation|-Umgebung. Diese wird genau wie die |equation|-Umgebung, �bernimmt aber als Parameter die anzuzeigende Formelnummer. (Der interne Z�hler f�r Formeln wird durch eine solche Gleichung nicht beeinflusst.
%
% \begin{center}
%  |\begin{refequation}{4.35c} a^2 + b^2 = c^2 \end{refequation}|
% \end{center}
% \begin{refequation}{4.35c}
%  a^2 + b^2 = c^2
% \end{refequation}
%
% \DescribeMacro{\sublabel}
% Dieser Befehl kann in einer Mathematik-Umgebung angewendet werden, in der Spalten m�glich sind. Es erzeugt an der entsprechenden Stelle eine Formelnummerierung, die der normalen folgt, allerdings pro Zeile immer die selbe Nummer verwendet und zus�tzlich einen kleinen Buchstaben anh�ngt (�hnlich zu |\begin{subequations}|). Um auf die Unterformel zu verweisen gibt man dem Label eine Identifikation: |\sublabel{eqn::erg1}|. Das Beispiel:
%
% %  |\begin{align}|\\
%  |a^2 &= b^2 + c^2 \sublabel{eqn::pyth} & | \\
%  |1 &= \sin^2 + cos^2 \sublabel{eqn::trig}|\\
%  |\label{eqn::beides}|\\
% |\end{align}|\\
% |\eqref{eqn::trig} folgt einfach aus \eqref{eqn::pyth}.|\\
% |Aus \eqref{eqn::beides} folgt $\ldots$|
%
% Erzeugt die Ausgabe:
% \begin{align}
%  a^2 &= b^2 + c^2 \sublabel{eqn::pyth} &
%  1 &= \sin^2 + cos^2 \sublabel{eqn::trig}
%  \label{eqn::beides}
% \end{align}
% \eqref{eqn::trig} folgt einfach aus \eqref{eqn::pyth}. Aus \eqref{eqn::beides} folgt $\ldots$
%
% \section{Chemische Befehle}
% \DescribeMacro{\mol}
% Um komplexe chemische Formeln zu schreiben gibt es sehr gute Pakete wie \textsf{ochem} und \textsf{mhchem}. Diese sind aber f�r die einfache Anwendung, wie sie hier gebraucht wird, zu komplex. Deshalb gibt es einen eigenen sehr einfachen Befehl, der sich um die richtige Schriftart k�mmert. Mit |\mol{}| kann man sehr einfache Formeln problemlos schreiben. Zum Beispiel erzeugt |\mol{H_2^+}| die Ausgabe \mol{H_2^+}.
%
% \StopEventually{\PrintChanges\PrintIndex}
% \section{Implementation}
%
% F�r die |\errnum|- und |\erSI|-Befehle ben�tigen wir zwei Pakete.
%    \begin{macrocode}
\RequirePackage{ifthen}
\RequirePackage{sistyle}
%    \end{macrocode}
% \begin{macro}{Differentialrechnung}
%    \begin{macrocode}
\newcommand\diff{\mathrm d}
\newcommand\dquot[3][]{\frac{\diff^{#1} #2}{{\diff #3}^{#1}}}
\newcommand\pdquot[3][]{\frac{\partial^{#1} #2}{{\partial #3}^{#1}}}
%    \end{macrocode}
% \end{macro}
% \begin{macro}{Integralrechnung}
%    \begin{macrocode}
\newcommand\Int{\int\limits}
\newcommand\dif{\,\diff}
%    \end{macrocode}
% \end{macro}
%
% \begin{macro}{Klammern}
%    \begin{macrocode}
\newcommand\kla[1]{\left(#1\right)}
\newcommand\brk[1]{\left[#1\right]}
\newcommand\brc[1]{\left\{#1\right\}}
\newcommand\abs[1]{\left|#1\right|}
%    \end{macrocode}
% \end{macro}
%
% \begin{macro}{Fehlerbehaftete Gr��en}
%    \begin{macrocode}
\newcommand\errnum[3][]{\ensuremath{%
 \ifthenelse{\equal{#1}{}}{%
  \num{#2} \pm \num{#3}%
 }{%
  (\num{#2} \pm \num{#3}) \cdot \num{#1}%
 }%
}}
\newcommand\errSI[4][]{\ensuremath{%
 \ifthenelse{\equal{#1}{}}{%
  (\num{#2} \pm \num{#3}) \; \SI{}{#4} %
 }{%
  (\num{#2} \pm \num{#3}) \cdot \SI{#1}{#4}%
 }%
}}
%    \end{macrocode}
% \end{macro}
%
% \begin{macro}{Diverse}
%    \begin{macrocode}
\newcommand\eu[1]{\mathrm e^{#1}}
%    \end{macrocode}
% \end{macro}
%
% \begin{macro}{refequation}
% Die |refequation|-Umgebung sichert den Wert des equation-Z�hlers sowie den |\theequation|-Befehl, �berschreibt letzteren, um das gew�nschte Verhalten mit einer einfachen |equation|-Umgebung zu erzielen. Am Ende werden der Z�hler und dessen Ausgabebefehl wiederhergestellt.
%    \begin{macrocode}
\newcounter{equationSAVE}
\newenvironment{refequation}[1]{%
 \setcounter{equationSAVE}{\value{equation}}%
 \let\backup@theequation\theequation%
 \edef\theequation{#1}%
 \begin{equation}%
}{%
 \end{equation}%
 \let\theequation\backup@theequation%
 \setcounter{equation}{\value{equationSAVE}}%
}
%    \end{macrocode}
% \end{macro}
%
% \begin{macro}{sublabel}
% Zun�chst werden Counter definiert, die sp�ter sicher stellen sollen, dass die Unternummerierung zur�ckgesetzt wird, wenn sich die Hauptformelnummer �ndert.
%    \begin{macrocode}
\newcounter{equationsave}
\setcounter{equationsave}0
\newcounter{lasteqn}
\setcounter{lasteqn}{-1}
\newcounter{subaligncnt}
%    \end{macrocode}
% Hier beginnt die eigentlich Definition. Zun�chst m�ssen die originalen Befehle von |\\| und |\end{align}| zwischengespeichert werden.
%    \begin{macrocode}
\newcommand\sublabel[1]{%
 \global\let\nextline\\
 \global\let\endalignsave\endalign
%    \end{macrocode}
% Der equation-Counter wird erst am Ende einer Zeile oder am Ende der Umgebung erh�ht. Er wird aber schon vorher von |sublabel| in der richtigen Z�hlung gebraucht.
%    \begin{macrocode}
 \setcounter{equationsave}{\theequation}
 \stepcounter{equationsave}
%    \end{macrocode}
% Wenn sich die Hauptnummerierung �ndert, muss die Unternummerierung zur�ckgesetzt werden. Richtig ist die Z�hlung: 1a, 1b, 2a (nicht 2c).
%    \begin{macrocode}
 \ifthenelse{\NOT\equal{\arabic{lasteqn}}{\arabic{equation}}}{%
  \setcounter{subaligncnt}1
 }{}
%    \end{macrocode}
% |\@currentlabel| enth�lt die Informationen der Nummerierung. Dazu wird die Hauptnummerirung genommen und ein kleiner Buchstabe f�r die Unternummerierung angeh�ngt.
%    \begin{macrocode}
 \def\@currentlabel{\arabic{equationsave}\ensuremath{\mathrm{\alph{subaligncnt}}}}
 \let\thepage\relax%
 \def\protect{\noexpand\noexpand\noexpand}%
%    \end{macrocode}
% Hier wird die eigentliche Information in die \emph{.aux}-Datei geschrieben, inklusive Alies, Nummerierung und Seitenzahl.
%    \begin{macrocode}
 \edef\@tempa{\write\@auxout{\string\newlabel{#1}{{\@currentlabel}{\thepage}}}}%
 \expandafter%
 \@tempa%
%    \end{macrocode}
% Das Label soll auch in der Formelausgabe erscheinen. Dazu wird das |\@currentlabel| expandiert in |\@ausgabe| geschrieben.
%    \begin{macrocode}
\global\edef\@ausgabe{\ensuremath{\quad(\@currentlabel)}}%
 &&\@ausgabe\nonumber
%    \end{macrocode}
% Oft ist es erw�nscht, auf alle Formeln zu verweisen, die in einer Zeile stehen. Dazu kann man am Ende (vor dem Zeilenumbruch) den bekannten |\label{alies}| Befehl verwenden. Da er aber von |\nonumber| �berschrieben wird, wird er an dieser Stelle neu definiert.
%    \begin{macrocode}
 \def\label##1{%
  \def\protect{\noexpand\noexpand\noexpand}%
  \edef\@tempa{\write\@auxout{\string\newlabel{##1}{{\arabic{equationsave}}{\thepage}}}}%
  \expandafter%
  \@tempa%
 }%
%    \end{macrocode}
% Bei einem Zeilenumbruch und bei |\end{align}| muss der Formelcounter erh�ht werden, da |\nonumber| normalerweise die Erh�hung verhindert.
%    \begin{macrocode}
 \def\endalign{\stepcounter{equation}\endalignsave}
 \def\\{\nextline\stepcounter{equation}}
 \stepcounter{subaligncnt}
%    \end{macrocode}
% Zur Pr�fung wird nun |lasteqn| auf den Wert von |equation| gesetzt.
%    \begin{macrocode}
 \setcounter{lasteqn}{\value{equation}}
}
%    \end{macrocode}
% \end{macro}
%
% \begin{macro}{Chemische Formeln}
%    \begin{macrocode}
\newcommand\mol[1]{\ensuremath{\mathrm{#1}}}
%    \end{macrocode}
% \end{macro}
%
% \Finale