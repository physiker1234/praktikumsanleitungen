\documentclass[platz]{tudphygp}
\usepackage{tudphymd,mhchem}

\versuch{Polarisation}{PO}

\begin{document}
\maketitle

\section*{Aufgabenstellung}

\begin{enumerate}
 \item Bestimmen Sie mit einem Polarimeter die Drehrichtung und die spezifische Drehung $\alpha_D$ f�r einen Quarz-Kristall aus 10 
 Einzelmessungen. Zur Bestimmung der Drehrichtung ist eine Gl�hlampe, f�r die spezifische Drehung $\alpha_D$ eine Spektrallampe (\ce{Na}, 
 $\lambda_{D-Linien} = (589,0 \pm 0,5)~\mathrm{nm}$) zu verwenden. Die Dicke $d$ der Quarz-Kristalle entnehmen Sie folgender Tabelle:
 
 \begin{center}
 \begin{tabular}{|c|c|}
 \hline
 Kennzeichnung	&	$d$ [mm]	\\
 \hline
 I				&	0,99 \\
 II				&	4,00 \\
 III			&	5,00 \\
 IV				&	2,50 \\
 \hline
 \end{tabular}
 \end{center}
 
 Fehler der Dicke $d$: $\left| \triangle d \right| = 0,01~\mathrm{mm}$.
 \item Ermitteln Sie die Drehrichtung und die Konzentration $c$ einer Saccharose - L�sung aus 10 Einzelmessungen mit einem Polarimeter.
 
 Die spezifische Drehung der Zuckerl�sung betr�gt: $\alpha_D = 6,65 \cdot 10^{-4} ~\frac{\mathrm{m}^2}{\mathrm{g}}$.
 \item Bestimmen Sie die Verdetsche Konstante $V$ von Toluol! Welches Vorzeichen hat $V_{Toluol}$?
\end{enumerate}

\section*{Hinweise}

\begin{itemize}
 \item[Zu 1./2.:] Die Quarze-Kristalle bzw. die K�vette sind vor jeder einzelnen Messung neu einzulegen und der Nullpunkt zu bestimmen.
 \item[Zu 3.:] Ermitteln Sie im Vorversuch die Apparaturkonstante $C$ der Zylinderspule unter Verwendung der bekannten Verdetschen Konstanten 
 $V$ von destilliertem Wasser: $V_{\ce{H2O}} = 0,1083 \cdot 1'~\mathrm{A}^{-1}$ ($1'~\mathrm{A}^{-1} \equiv$ Winkelminuten pro Ampere). $C$ 
 folgt aus der Abh�ngigkeit des Drehwinkels der Polarisationsebene $\alpha$ von der Stromst�rke $I$ (linearer Zusammenhang).
 
 \textbf{Achtung:} $\left| I_{max} \right| = 4~\mathrm{A}$
\end{itemize}

\end{document}