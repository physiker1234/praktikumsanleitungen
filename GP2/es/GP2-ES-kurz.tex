\documentclass[platz]{tudphygp}
\usepackage{tudphymd,mhchem}

\versuch{Erzwungene Schwingungen}{ES}

\begin{document}
\maketitle

\section*{Aufgabenstellung} 

\begin{enumerate}
 \item Bestimmung der Eigenfrequenz $\omega_0$ der unged�mpften Schwingung sowie der 
 D�mpfungsfaktoren $\delta$ f�r verschiedene $B$-Werte der Wirbelstrombremse.
 \item Messung der Frequenzabh�ngigkeit von Amplitude $\hat\varphi(\omega)$ und Phase $\alpha(\omega)$ 
 mit Hilfe des Pohlschen Rades im Bereich von $f=\num{0,2}$ bis $\SI{1,2}{Hz}$ f�r drei verschiedene
 D�mpfungen $\left(\delta\approx\frac{\omega_0}{4}~;~\frac{\omega_0}{8}~;~\frac{\omega_0}{20}\right)$.
 \item Vergleich a.) des G�te-Wertes und der Bandbreiten mit der theoretischen N�herung und b.) einer 
 auf den Maximal-Wert normierten experimentellen Resonanzkurve mit der berechneten (in der Vorbereitung f�r
 $\omega_0=4\delta$ sowie $\omega_0=\SI{3,14}{Hz}$).
\end{enumerate}

\section*{Hinweise zur Versuchsdurchf�hrung}


\subsection*{Ger�teparameter}


\end{document}
