\documentclass[platz]{tudphygp}
\usepackage{tudphymd,mhchem}
\usepackage{xcolor}

\versuch{Brechzahl und Dispersion}{DP}

\begin{document}
\maketitle

\section*{Aufgabenstellung}

\begin{enumerate}
 \item Mit dem Refraktometer nach \textsc{Abbe} sind mit wei�em Licht die Brechzahlen $n_D$ und die
 Standard-Dispersion $n_F-n_C$ von destilliertem Wasser, einer zweiten Fl�ssigkeit und
 einer Glasplatte zu bestimmen.
 \item Es werden Brechzahl $n_D$ und Standard-Dispersion $n_F-n_C$ f�r den ordentlichen und
 den au�erordentlichen Strahl eines Quarz-Kristalls bestimmt.
 \item F�r destilliertes Wasser ist bei Beleuchtung mit einer \ce{Cd}-Spektrallampe und einer \ce{Hg}-
 Spektrallampe die Abh�ngigkeit der scheinbaren Brechzahl $x$ von der Wellenl�nge $\lambda$
 zu messen und daraus mittels Korrekturrechnungen die Dispersionskurve $n(\lambda)$ zu
 bestimmen. Anschlie�end ist die G�ltigkeit der von \textsc{Kislovskii} (1959) f�r Wasser
 angegebenen Dispersions-Formel
 
 \begin{equation*}
 n(\lambda) = \left[ 1,29 + \frac{0,47 \cdot {\lambda}^2}{{\lambda}^2 - (0,119~\mathrm{\mu m})^2} - 
  \frac{0,08 \cdot {\lambda}^2}{(2,92~\mathrm{\mu m})^2 - {\lambda}^2} \right]^{\frac{1}{2}}
  \quad \text{mit} \quad \lambda ~ \text{in} ~ \mathrm{\mu m}
 \end{equation*}
 
 zu pr�fen. Tragen Sie dazu in die grafische Darstellung des aus obiger Gleichung
 errechneten $n(\lambda)$-Zusammenhangs die aus der Messung folgenden $n(\lambda)$-Werte ein und
 vergleichen Sie diese Werte miteinander.
\end{enumerate}

\section*{Hinweise}

\begin{itemize}
 \item Das Messprisma des Refraktometers ist sehr vorsichtig zu behandeln; bitte nach jeder
 Messung und abschlie�end mit Zellstoff und Alkohol sowie destilliertem Wasser
 reinigen!
 \item \textbf{Zu 1.:} Nach Einstellung der Beleuchtung mit der Gl�hlampe und Verdrehen des kleinen
 Hilfsspiegels zur Aufhellung der linken inneren Skale wird das Prisma vorsichtig
 ge�ffnet. Mit dem glatten Glasstab werden zwei Tropfen Wasser auf das horizontal
 gestellte Prisma gebracht und dieses verschlossen. Nach dem Drehen des Prismas am
 linken Knopf wird der in Einheiten von $n_D$ geeichte Winkel links abgelesen. Dies gelingt
 nur, wenn vorher der Farbsaum mit dem Kompensator (rechter Drehknopf;
 Geradsichtprismen-Paar nach Amici) beseitigt und scharf auf hell-dunkel abgeglichen
 wurde. Die dazu notwendige Stellung des Kompensators wird auf der rechten Skala
 abgelesen und notiert.
 \item Zur Bestimmung der Dispersion wird mit dem notierten Kompensator-Wert (z.B. $z=40$)
 unter Zuhilfenahme des $n_D$-Wertes nach der Hersteller-Betriebsanleitung verfahren.
 \item \textbf{Zu 1. und 2.:} Bei der Messung der Brechzahl $n_D$ der Glasplatte und des Quarzkristalls
 bleibt das Prisma ganz aufgeklappt und die Glasplatte bzw. die Quarzkristall-Scheibe
 wird mit einem Tropfen Monobromnaphtalin (\ce{C10H7Br}; $n=1,6572$) auf das Messprisma
 geklebt. Der streifende Einfall der Beleuchtung wird durch die matten Seitenfl�chen der
 K�rper (u.U. auch mit diffusem Tageslicht) erreicht. Danach wird analog zu 1. weiter
 verfahren. \\
 \textbf{Am Ende bitte das Prisma mit Zellstoff, Alkohol und dest. Wasser
 vorsichtig reinigen!}
 \item \textbf{Zu 3.:} Bei festgehaltener Kompensatorstellung (Null entspr. $z=30$) werden bei
 Verwendung der \ce{Cd}-Spektrallampe die Grenzen der farbigen Felder ins Fadenkreuz des
 Refraktometers gebracht (rechtes Okular) und im linken Okular die scheinbaren
 Brechzahlen $x$ an der linken Skala abgelesen. Die Spektrallinien der \ce{Hg}-Lampe sind
 mittels geeigneter Metallinterferenzfilter auszufiltern, so dass die zugeordnete
 scheinbare Brechzahl $x$ anhand des zu beobachtenden Farbe-Schwarz-�bergangs zu
 ermitteln ist.
 \item Die Umrechnung der scheinbaren Brechzahlen $x(\lambda)$ auf die Brechzahlen $n(\lambda)$ erfolgt
 nach Gl. (\ref{gl1}). Die dazu ben�tigten Brechzahlen des Messprismas $n^{'}$ und der
 Prismenwinkel $\alpha$ sowie die Formel zur Berechnung von sind (Gl.(\ref{gl2})) sind unten
 angegeben.
\end{itemize}

\textbf{Wesentliche Beziehungen:}

\begin{equation} \label{gl1}
n = \sin \alpha \cdot \sqrt{{n^{'}}^2 - {\sin}^2 \delta} - \cos \alpha \cdot \sin \delta
\end{equation}
\begin{equation} \label{gl2}
\sin \delta  = \sin \alpha \cdot \sqrt{{{n_D}^{'}}^2 - x^2} - x \cdot \cos \alpha
\end{equation}

\textbf{Kennwerte des Messprismas:}

Die Brechzahlen $n^{'}(\lambda)$ des Messprismas k�nnen unter Anwendung der \textsc{Cauchy}-Beziehung

\begin{equation}
n^{'} (\lambda) = \left( A_0^{'} + \frac{A_1^{'}}{{\lambda}^2} + \frac{A_2^{'}}{{\lambda}^4} \right)^{\frac{1}{2}}
\end{equation}

mit

\begin{tabular}{rlll}
neues Refraktometer, F9: & $A_0^{'}=2,88959$ & $A_1^{'}=0,04892~\mathrm{\mu m}^2$ & $A_2^{'}=0$ \\
altes Refraktometer, F7: & $A_0^{'}=2,91277$ & $A_1^{'}=0,03439~\mathrm{\mu m}^2$ & $A_2^{'}=0,00220838~\mathrm{\mu m}^4$
\end{tabular}

berechnet werden (bis 5 Stellen nach dem Komma verwenden!).

\textbf{Prismenwinkel:}

$\alpha=62,937 � \quad \text{(neues Refraktometer, F9)}$\\
$\alpha=62,945 � \quad \text{(altes Refraktometer, F7)}$

\textbf{Vakuum-Wellenl�ngen der \ce{Hg}-Lampe, der \ce{Cd}-Lampe und dreier Fraunhofer-Linien:}

{\renewcommand{\arraystretch}{1.1}

\begin{center}
\begin{tabular}{cc}

\ce{Hg}-Spektrum: & \ce{Cd}-Lampe: \\

\begin{tabular}{|c|c|c|}
\hline
Linie & Farbe & $\lambda~[\mathrm{\mu m}]$ \\ 
\hline
1  & \textcolor[rgb]{1.0,0.2,0.2}{\textbf{rot schwach}} & 614,4 \\ 
2  & \textcolor[rgb]{0.89,0.58,0.58}{\textbf{rot sehr schwach}} & 610,0 \\ 
3  & \textcolor[rgb]{0.89,0.58,0.58}{\textbf{rot sehr schwach}} & 608,4 \\ 
4  & \textcolor[rgb]{1.0,1.0,0.0}{\textbf{gelb stark}} & 579,1 \\ 
5  & \textcolor[rgb]{1.0,1.0,0.0}{\textbf{gelb stark}} & 577,0 \\ 
6  & \textcolor[rgb]{0.2,0.8,0.2}{\textbf{gr�n sehr stark}} & 546,1 \\ 
7  & \textcolor[rgb]{0.0,0.96,0.96}{\textbf{blaugr�n schwach}} & 498,1 \\ 
8  & \textcolor[rgb]{0.2,0.6,1.0}{\textbf{blaugr�n mittel}}  & 491,6 \\ 
9  & \textcolor[rgb]{0.6,0.2,1.0}{\textbf{violett stark}} & 435,8 \\ 
10  & \textcolor[rgb]{0.8,0.6,1.0}{\textbf{violett schwach}} & 407,9 \\ 
11  & \textcolor[rgb]{0.7,0.4,1.0}{\textbf{violett mittel}} & 404,7 \\
\hline 
\end{tabular} 

&

\begin{tabular}{c}

\begin{tabular}{|c|c|c|}
\hline
Linie & Farbe & $\lambda~[\mathrm{\mu m}]$ \\ 
\hline
1  & \textcolor[rgb]{1.0,0.2,0.2}{\textbf{rot schwach}} & 643,8 \\ 
2  & \textcolor[rgb]{0.2,0.8,0.2}{\textbf{gr�n sehr stark}} & 508,6 \\ 
3  & \textcolor[rgb]{0.0,0.8,1.0}{\textbf{blaugr�n stark}} & 480,0 \\ 
4  & \textcolor[rgb]{0.2,0.6,1.0}{\textbf{blaugr�n mittel}} & 467,8 \\ 
\hline
\end{tabular} 
\vspace{0.05\linewidth} \\
Fraunhofer-Linien:\\
\begin{tabular}{|c|c|c|}
\hline
Linie & Farbe & $\lambda~[\mathrm{\mu m}]$ \\ 
\hline
F  & \textcolor[rgb]{0.0,0.8,1.0}{\textbf{blaugr�n}} & 486 \\ 
C  & \textcolor[rgb]{1.0,0.0,0.0}{\textbf{rot}} & 656 \\ 
D  & \textcolor[rgb]{1.0,1.0,0.0}{\textbf{gelb}} & 589 \\  
\hline
\end{tabular}
\end{tabular}

\end{tabular}
\end{center}
}

\end{document}