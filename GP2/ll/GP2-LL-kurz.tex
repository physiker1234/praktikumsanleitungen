\documentclass[platz]{tudphygp}
\usepackage{tudphymd,mhchem}

\versuch{Lecherleitung}{LL}

\begin{document}
\maketitle

\section*{Aufgabenstellung} 
\begin{enumerate}
 \item Bestimmen Sie mit der in der Versuchsvorbereitung angegebenen Messanordnung die Wellenl�nge $\lambda_{Luft}$ der stehenden 
 elektromagnetischen Welle im Lechersystem in Luft und sch�tzen Sie den Fehler ab.
 \item Berechnen Sie mit Hilfe des Ergebnisses aus der Aufgabe 1 die Frequenz des Hochfrequenz-Senders, der das Lechersystem speist.
 \item Bestimmen Sie in analoger Art der Aufgabe 1 die Wellenl�nge $\lambda_{Wasser}$ der stehenden elektromagnetischen Welle im 
 destillierten Wasser und sch�tzen Sie den Fehler ab.
 \item Berechnen Sie mit Hilfe des Ergebnisses aus der Aufgabe 3 die relative Dielektrizit�tszahl $\epsilon_r$ von Wasser und diskutieren 
 Sie das Ergebnis im Vergleich zu Tabellenwerten.
\end{enumerate}

\section*{Hinweise}
\begin{enumerate}
 \item Das Lechersystem ist so aufgebaut, dass Sie mit der angegebenen Methode in Luft mindestens drei Maxima sehr genau aufl�sen k�nnen. 
 Machen Sie daher erst eine Grobmessung und verringern Sie die Messst�nde im Bereich der Maxima auf die kleinsten sinnvollen Werte.

 \item Da das Lechersystem Energie abstrahlt, ist eine definierte Position des Experimentators und eine �berlegte F�hrung der Messleitung 
 notwendig, da sonst ihre Ergebnisse verf�lscht werden.
\end{enumerate}

\end{document}
