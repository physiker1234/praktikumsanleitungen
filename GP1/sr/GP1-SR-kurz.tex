\documentclass[platz]{tudphygp}
\usepackage{mhchem,tudphymd}

\versuch{Str�mung im Rohr}{SR}


\begin{document}
\maketitle

\section*{Aufgabenstellung}

\begin{enumerate}
 \item Bestimmen Sie den Druckabfall (indirekt �ber $\Delta h$) als Funktion der
 Str�mungsgeschwindigkeit von Wasser in einem horizontalen Rohr.
 \item Berechnen Sie die Widerstandsbeiwerte ($c_w$) und Reynoldschen Zahlen ($Re$) im laminaren und turbulenten Gebiet, tragen Sie $c$ �ber $Re$ doppelt logarithmisch auf und und vergleichen Sie mit den theoretischen Werten.
 \item Bestimmen Sie die kritische Reynoldszahl $Re_\mathrm{krit}$, diskutieren Sie das Ergebnis.
 \item F�hren Sie die Fehlerbetrachtung f�r $c_w$ und $Re$ f�r zwei weit auseinanderliegende Punkte des Diagramms durch.
\end{enumerate}

\section*{Hinweise zur Versuchsdurchf�hrung}


\begin{itemize}
\item Die g�nstigste Ausgangsh�he der Fl�ssigkeit wird so eingestellt, dass sich die Wassers�ulen in Augenh�he befinden. Beachten Sie, dass ein H�henunterschied von bis zu \SI{300}{mm} zu messen ist.
%
\item Im Bereich niedriger Druckdifferenzen sind m�glichst
viele Werte zu messen. Abwarten, bis der jeweilige $\Delta h$-Wert
konstant ist. Verwenden Sie folgende Messpunkte:
\[
 \Delta h = (1, 2, 3, 4, 5, 6, 8, 10, 12, 15, 20, 50, 100, 200, 300) \SI{}{mm}
\]
\item Eingedrungene Luftblasen m�ssen (durch Klopfen) entfernt werden.
\item F�r die Aufnahme der Messwerte eignet sich Tabelle \ref{tab::messwerte}.
\begin{table}[b]
 \centering
 \begin{tabular}{|c|c|c|c|c|c|c|}\hline
  $\Delta h$/\SI{}{mm} & $V$ / \SI{}{ml} & $t$/\SI{}s & $\theta$ / \SI{}\degC & $\eta/\rho$ / \SI{}{cm^2/s} & $Re$ & $c_w$\\\hline
  $\vdots$&&&&&&\\\hline
 \end{tabular}
 \caption{Beispiel Messwertetabelle \label{tab::messwerte}}
\end{table}
%
\item Berechnen Sie zuerst die Konstanten ($R$ ist der Rohrradius und $l$ die L�nge der Messstrecke):
 \begin{align*}
  K_{Re} &= \frac2{\pi R} & [K_{Re}] &= \SI{}{cm^{-1}}\\
  K_c &= \frac1l\cdot \pi^2 g\cdot R^5 & [K_c] &= \SI{}{cm^2/s^2}
 \end{align*}
%
\item Sie ben�tigen folgende Formeln:
\[
 c_w = K_c\cdot \frac{\Delta h}{\kla{\frac Vt}^2} \quad\text{und}\quad Re = \frac{K_{Re}}{\eta/\rho}\cdot\kla{\frac Vt}
\]
\item Die temperaturabh�ngige kinematische Z�higkeit von Wasser folgt der Formel:
\[
 %TODO: evtl. polynomieller statt exponentieller Fit
 \frac\eta\rho(\theta) = \num{1,726}\cdot \eu{-\num{4,605}+\num{0,028}\theta /\SI{}\degC}\,\SI{}{cm^2/s}
\]
\end{itemize}

\clearpage

\section*{Ger�teparameter}

\begin{table}[h]
 \centering
 \begin{tabular}{|c|c|c|c|c|}\hline
  Messplatz & Radius $R$ /\SI{}{mm} & $\Delta R$ / \SI{}{mm} & L�nge $l$ / \SI{}{mm} & $\Delta l$ / \SI{}{mm}\\\hline
  a & \num{3,5} & \num{,1} & \num{199,6} & \num{,5} \\
  b & \num{2,5} & \num{,1} & \num{59,2} & \num{,5} \\
  c & \num{5} & \num{,1} & \num{480,2} & \num{,5} \\
  d & \num{5} & \num{,1} & \num{289,1} & \num{,5} \\\hline
 \end{tabular}
 \caption{Daten der Messaparaturen}
\end{table}

\end{document}
