\documentclass[platz]{tudphygp}
\usepackage{tudphymd}

\versuch{Luftfeuchtigkeit}{LF}

\begin{document}
\maketitle

\subsection*{Aufgabenstellung}

Bestimmen Sie die absolute und relative Luftfeuchtigkeit mithilfe eines Aspirationspsychrometers nach A�mann.

\subsection*{Durchf�hrung}

F�r ein station�res thermodynamisches Gleichgewicht gilt ensprechend der Anleitung:
\begin{refequation}{7}
 p_\mathrm w = p_\mathrm{w,s}(T_\mathrm w) - p_\mathrm L \cdot (T - T_\mathrm w) \cdot A^* \quad\text{mit}\quad A^* = \frac{c_p}{q_v} \cdot \frac{M_\mathrm L}{M_\mathrm w}
\end{refequation}%
Es kann aber im Freien (z.B. durch pl�tzliche Sonneneinstrahlung, Strum, etc.) oder in R�umen (z.B. durch Zugluft, Heizung, etc.) zu kurzfristigen thermodynamischen �nderungen kommen. Im Versuch ist es anzustreben, mindestens \num{15} Minuten unter konstanten Bedinungen zu messen und �nderungen zu notieren. Die mittlere molare Masse $M_\mathrm L$ der trockenen Luft ergibt sich aus den vier Hauptbestandteilen \mol{N_2}, \mol{O_2}, \mol{CO_2} und \mol{Ar} zu $M_\mathrm L = \SI{28,954}{\frac g{mol}}$ (wie?). Wasser hat die molare Masse $M_\mathrm w = \SI{18,00}{\frac g{mol}}$. F�r $c_p$ setzt man den N�herungswert $c_p = \SI{1,01}{\frac J{g.K}}$ ein und begr�ndet Abweichungen in der Fehlerrechnung (siehe Anleitung, Abschnitt 4.3). $q_v$ ist temperaturabh�ngig (siehe Tab.~\ref{tab::warmeKap}). F�r $\vartheta < 0\degC$ nutzt man die spezifische Sublimationsw�rme von Eis $q_s = \SI{2850}{\frac Jg}$.

\begin{enumerate}
 \item Im trockenen Zustand werden die Thermometer zur Kontrolle abgelesen, die L�fter eingeschaltet und beide Thermometer noch einmal abgelesen. Danach wird das feuchte Thermometer durch einmaliges Eintauchen in destilliertes Wasser angefeuchtet. Nach Einstellung eines station�ren Zustandes (Temperaturverlauf beobachten und protokollieren) wird �ber einen Zeitraum von 10 bis 15 Minuten je einmal pro Minute $T$ und $T_\mathrm w$ mit einer Genauigkeit von \SI{0,1}K gemessen.
 \item Zur Kontrolle wird das feuchte Thermometer erneut angefeuchtet. Nach anf�nglichem Temperaturanstieg (Begr�ndung!) muss sich der gleiche Endwert einstellen.
 \item Die L�fter sind so lange in Betrieb zu lassen, bis beide Thermometer wieder die gleiche Temperatur anzeigen, also das feuchte Thermometer getrocknet ist.
\end{enumerate}

\begin{table}[p]\centering
 \begin{tabular}{cc|cc|cc|cc}
  $\vartheta/\degC$ & $q_v/\SI{}{J\,g^{-1}}$ & $\vartheta/\degC$ & $q_v/\SI{}{J\,g^{-1}}$ & $\vartheta/\degC$ & $q_v/\SI{}{J\,g^{-1}}$ & $\vartheta/\degC$ & $q_v/\SI{}{J\,g^{-1}}$ \\\hline
  0              & 2525                   & 8              & 2483                   & 16             & 2464 & 24  & 2445 \\
  1              & 2499                   & 9              & 2480                   & 17             & 2461 & 25  & 2443 \\
  2              & 2497                   & 10             & 2478                   & 18             & 2459 & 26  & 2440 \\
  3              & 2494                   & 11             & 2476                   & 19             & 2457 & 27  & 2438 \\
  4              & 2492                   & 12             & 2473                   & 20             & 2454 & 28  & 2435 \\
  5              & 2490                   & 13             & 2471                   & 21             & 2452 & 29  & 2433 \\
  6              & 2487                   & 14             & 2469                   & 22             & 2450 & 30  & 2433 \\
  7              & 2485                   & 15             & 2466                   & 23             & 2447 & 100 & 2249 \\
 \end{tabular}
 \caption{Temperaturabh�ngige Verdampfungsw�rme von Wasser} \label{tab::warmeKap}
\end{table}

\begin{table}[p]\centering
 \begin{tabular}{ccc|ccc}
  $\vartheta / \degC$ & $p_{W,S}/\SI{}{Pa}$ & $\varrho_{W,S} / \SI{}{g.m^{-3}}$ & $\vartheta / \degC$ & $p_{W,S} / \SI{}{Pa}$ & $\varrho_{W,S} / \SI{}{g.m^{-3}}$ \\\hline
  -10 & 260  & \num{2,14}  & 17 & 1938 & \num{14,47} \\
   -9 & 284  & \num{2,33}  & 18 & 2064 & \num{15,36} \\
   -8 & 310  & \num{2,33}  & 19 & 2198 & \num{16,30} \\
   -7 & 338  & \num{2,75}  & 20 & 2339 & \num{17,29} \\
   -6 & 369  & \num{2,99}  & 21 & 2487 & \num{18,33} \\
   -5 & 421  & \num{3,25}  & 22 & 2644 & \num{19,42} \\
   -4 & 437  & \num{3,52}  & 23 & 2810 & \num{20,57} \\
   -3 & 476  & \num{3,82}  & 24 & 2985 & \num{21,77} \\
   -2 & 517  & \num{4,14}  & 25 & 3169 & \num{23,04} \\
   -1 & 562  & \num{4,48}  & 26 & 3363 & \num{24,37} \\
    0 & 610  & \num{4,85}  & 27 & 3567 & \num{25,76} \\
    1 & 657  & \num{5,19}  & 28 & 3782 & \num{27,23} \\
    2 & 706  & \num{5,56}  & 29 & 4008 & \num{28,76} \\
    3 & 758  & \num{5,94}  & 30 & 4245 & \num{30,37} \\
    4 & 813  & \num{6,36}  & 31 & 4495 & \num{32,05} \\
    5 & 872  & \num{6,80}  & 32 & 4758 & \num{33,82} \\
    6 & 935  & \num{7,26}  & 33 & 5033 & \num{35,66} \\
    7 & 1002 & \num{7,75}  & 34 & 5323 & \num{37,59} \\
    8 & 1073 & \num{8,27}  & 35 & 5626 & \num{39,61} \\
    9 & 1148 & \num{8,82}  & 36 & 5945 & \num{41,72} \\
   10 & 1228 & \num{9,40}  & 37 & 6279 & \num{43,93} \\
   11 & 1313 & \num{10,01} & 38 & 6630 & \num{46,24} \\
   12 & 1403 & \num{10,66} & 39 & 6697 & \num{48,65} \\
   13 & 1498 & \num{11,34} & 40 & 7381 & \num{51,16} \\
   14 & 1599 & \num{12,06} & 45 & 9590 & \num{65,46} \\
   15 & 1705 & \num{12,82} & 50 & 12345 & \num{83,02} \\
   16 & 1818 & \num{13,63} &  &  &
 \end{tabular}
 \caption{Dampfdruck $p_{W,S}$ und Dampfdichte $\varrho_{W,S} = f_0$ des ges�ttigten Wasserdampfes \newline (aus: Kohlrausch, Bd.~3, 1996)} \label{tab::dampf}
\end{table}
\end{document}