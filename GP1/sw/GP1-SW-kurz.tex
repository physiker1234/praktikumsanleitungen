\documentclass[platz]{tudphygp}
\usepackage{mhchem,tudphymd}

\versuch{Stehende Wellen}{SW}


\begin{document}
\maketitle

\section*{Aufgabenstellung}

\begin{enumerate}
 \item Nach der Methode von Quincke ist anhand der Schallgeschwindigkeit der Luft die
 Frequenz $f$ zweier gegebener Stimmgabeln (Mittelwert von jeweils f�nf Messreihen) zu
 ermitteln.
 \item Mittels Tonfrequenzgenerator, Lautsprecher und Mikrophon ist mit Hilfe stehender
 Wellen in einem horizontalen Glasrohr (\textsc{Kundt}sches Rohr) nahe der Reflexionswand
 die Schallgeschwindigkeit $c$ in Gasen (Luft) zu bestimmen und auf \SI0{\degC} umzurechnen.
\end{enumerate}

\section*{Hinweise zur Versuchsdurchf�hrung}

Sie ben�tigen die Beziehung zwischen Wellenl�nge und Schallgeschwindigkeit $c = \lambda f$ und die Gr��e des Kontenabstandes: $d = \lambda/2$.

\subsection*{zu 1.}

\begin{itemize}
 \item Beachten Sie zum Einspannen der Stimmgabel und Inbetriebnahme des Generators die Bedienungsanleitung unten.
 \item Den Abstand der Stimmgabel m�glichst nicht ver�ndern, die Stimmgabel
 gegebenenfalls mit dem kleinen Hammer ansto�en.
 \item Mindestens zwei Stellungen (Knotenpositionen) der Wasseroberfl�che im
 Rohr, bei denen das Quincke-Rohr
 in Resonanz f�r den Grundton mit maximaler Lautst�rke ger�t, durch zehnmaliges Ausmessen bestimmen.
 \item Man darf sich durch m�gliche, bei zu starker R�ckkopplung mit angeregte
 Oberschwingungen nicht beirren lassen.
\end{itemize}

\subsection*{zu 2.}
\begin{itemize}
 \item  Das Glasrohr ist an einer Seite mit einem Lautsprecher, an der anderen mit
 einer Wand abgeschlossen. Durch die Wand wird ein Mikrophon in das Rohr
 eingef�hrt. Der Lautsprecher wird mit einem RC-Generator angeregt. Im Resonanzfall
 bilden sich im Rohr stehende Wellen aus. Der Verlauf des Schalldruckes 
 wird �ber der Rohrl�nge mit dem Mikrophon gemessen. Es werden
 drei Frequenzen $f$ zwischen \num{2,5} und \SI{3,5}{kHz}
 benutzt. Fehler der Frequenz ist kleiner als \SI1\%. Der Effektivwert der Spannung sollte etwa \num{0,06} bis \SI{0,08}V betragen.
 \item Das batteriebetriebene Mikrophon, das auf den Schallwechseldruck reagiert,
 muss nach \SI{30}{min} mit dem roten Knopf wieder eingeschaltet werden (Abschaltautomatik).
%  ; Stellung = benutzen. HINWEIS: ???
 Das Mikrophon wird an ein Multimeter im \SI2V-Messbereich angeschlossen. Die Empfindlichkeit des Mikrophons
 ist zwischen \num{18} und \SI{300}{mV/mbar} einstellbar. Durch langsames (in \SI{}{mm}-Abst�nden)
 Verschieben der Mikrophon-Achse in L�ngsrichtung k�nnen mehrere Knoten
 und B�uche ausgemessen werden (mindestens �ber drei Halbwellen). Der
 Mikrophontransport erfolgt von Hand auf einem Transportwagen.
 \item Durch grafische Darstellung der Spannung in der N�he der Knoten kann die
 Knotenposition mittels Extrapolation genauer bestimmt werden. In der N�he
 der Wand werden Knoten und B�uche ausgemessen.
 \item Nach Beendigung der Messung mit der jeweiligen Frequenz $f$ sollte diese
 nachtr�glich pr�ziser mit einem Frequenz-Z�hler, dem Strahlungsmessger�t 20046, ermittelt werden.
 Dazu ist die Spannung des Tonfrequenzgenerators auf \SI2V zu erh�hen, der Pegel des Diskriminators
 auf \num{1,0} zu setzen, was etwa \SI10{\%} entspricht. Der zweite linke Knopf auf wird auf "`Dis."' gestellt,
 der Verst�rker auf \SI{16}{db} und die Impulsvorwahl auf \num{e5}.
 Alle rechts befindlichen Kn�pfe sind in die �ussere Stellung zu bringen.
 \item Arbeiten Sie entweder mit Zeitvorwahl (z.~B.~\SI{40}s), wobei die Impulszahl angezeigt wird, oder
 Impulsvorwahl (z.~B.~\num{e5}) wobei die Zeit angezeigt wird. Bei letzterer Version hat man einen besseren
 Einblick in die Genauigkeit der Zeitmessung.
\end{itemize}

\section*{Bedienung des Stimmgabelverst�rkers}

\begin{description}
 \item [Einspannen der Gabel] Der Erregerkopf wird in die h�chstm�gliche Lage gebracht und die Stimmgabel
 m�glichst kurz eingespannt. Danach ist der Erregerkopf parallel im Abstand von
 ca. \SI1{mm} so anzuordnen, dass er genau mit einem freien Gabelende abschlie�t.
 \item [Inbetriebnahme des Generator] Schlie�en Sie den Netzstecker an und schalten Sie den Generator �ber den
 Kippschalter ein, die Netzkontrolle sollte nun leuchteen. Nach der Anlaufzeit des Generators beginnt die
 Stimmgabel zu schwingen. Stellen Sie zuletzt den Spalt zwischen Erregerkopf und Stimmgabel auf optimale Lautst�rke minimal ein.
 \item [Achtung!] Betreigen Sie den Generator immer bei ausreichendem Abstand zwischen Stimmgabel und
 Erregerkopf, um ein Mitschwingen der Stimmgabelaufh�ngung und �berlas-
 tungen zu vermeiden.
\end{description}

\end{document}
