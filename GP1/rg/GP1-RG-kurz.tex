\documentclass[platz]{tudphygp}
\usepackage{tudphymd}

\versuch{Reale Gase}{RG}

\begin{document}
\maketitle

\subsection*{Aufgabenstellung}

\begin{enumerate}
 \item Messen Sie vier Isothermen eines realen Gases 
       (Schwefelhexafluorid \mol{SF_6}).
 \item Bestimmen Sie das Koexistenzgebiet von gasf\"ormiger und fl\"ussiger Phase und den kritischen Punkt.
 \item Bestimmen Sie aus den Messdaten die Stoffmenge und die
       van-der-Waals-Koeffizienten. Berechnen Sie das molare Volumen f\"ur den kritischen Punkt.
 \item \emph{(F\"ur Physikstud.)} Bestimmen Sie aus der Dampfdruckkurve $p_S(T)$
                                die molare Verdampfungsw\"arme und zu jedem
                                Messpunkt mit $T < T_c$ die Verdampfungsw\"arme $Q$.
\end{enumerate}

\subsection*{Hinweise zur Versuchsdurchf\"uhrung}

Der Versuch wird mit einer Apparatur zum kritischen Punkt
durchgef\"uhrt. Mittels einer Spindelmechanik und eines Kolbens wird ein durch
eine Gummidichtung abgetrenntes Volumen Schwefelhexaflourid (\mol{SF_6}) innerhalb
eines transparenten Zylinders komprimiert. Als Druck \"ubertragendes Medium
kommt Rizinus\"ol zum Einsatz. Die gesamte Anordnung wird durch einen
Umw\"alzthermostaten thermisch stabilisiert. 

Der PC am Versuchplatz dient zur teilweise automatisierten
Messwertaufnahme. Es werden der Spindelweg bzw. das Volumen, sowie die
Temperatur der Apparatur und der \"Oldruck gemessen und ein $p$-$V$-Diagramm
erstellt. 

\begin{description}
 \item [Achtung!] Der Maximaldruck von \SI{5}{MPa} (\SI{50}{bar}) darf auf
   keinen Fall \"uberschritten werden. Die Temperatur darf \SI{60}{\degC} nicht
   \"ubersteigen. Nach Beendigung des Versuches ist ein Druck von etwa \SI1{MPa}
   (\SI{10}{bar}) einzustellen. Vermeiden Sie l\"angere Standzeiten bei hohem Druck!
\end{description}

Die Isothermen sollen bei \SI{30}{\degC}, \SI{38}{\degC}, \SI{45,5}{\degC} und
\SI{55}{\degC} aufgenommen werden. Warten Sie nach dem Verstellen des
Thermostaten 
immer ca. \SI{10}{min} bis zum Erreichen einer S\"attigungstemperatur, sie
weicht meist etwas von der Vorgabe ab. Die kritische Temperatur soll so genau
wie m\"oglich eingestellt werden. Die anderen Temperaturen sind Richtwerte. Geringe
Temperaturschwankungen im Bereich weniger Zehntel Kelvin w\"ahrend einer
Messreihe sind tolerierbar.

Stellen Sie den Kolben auf genau \SI{42}{mm} ein (Skala am Ger\"at), bevor Sie eine
Messung starten, damit das Volumen richtig berechnet wird.

Starten Sie jetzt das Programm \"uber die Verkn\"upfung ``RG'' auf dem
Desktop. Warten Sie, bis das Messger\"at erkannt und das Programm initialisiert
wurde. Bl\"attern Sie auf Seite 2 der Oberfl\"ache und starten Sie die
Messung. Jetzt werden die aktuellen Messwerte angezeigt.

Beginnen Sie mit der Messung etwa bei $V=\SI{8}{ml}$ und verringern Sie das
Volumen anfangs in Schritten von \SI{1,0}{ml}. Zu Beginn und Ende des
Koexistenzgebietes sollte die Schrittweite auf \SI{0,25}{ml} verringert
werden, um den kritischen Punkt herum auf \SI{0,1}{ml}. Messen Sie die
Isothermen bis ca. $p = \SI{45}{bar}$.

Die \"Ubernahme des Messwertes erfolgt durch den Befehl ``Behalten'', der
Abschluss einer Messreihe durch ``Stoppen''. Vor jedem Messpunkt muss anfangs
ca. $30\ldots\SI{60}{s}$, sp\"ater bis zu \SI2{min} gewartet werden, bis sich
das thermische Gleichgewicht eingestellt hat (stabiler Druck). 

Notieren Sie zu jeder Isotherme die ersten Wertepaare f\"ur $V\ge \SI4{ml}$ in
einer Tabelle. Nach Abschluss der Messreihen kann das Diagramm ausgedruckt
werden (``Drucken'' $\to$ Seite 2, ``Allgemein'' $\to$ Querformat). 


\subsection*{Auswertung}

\begin{enumerate}
 \item Beschreiben Sie Ihre Beobachtungen der Messzelle beim Durchfahren des Koexistenzgebietes.
 \item In dem ausgedruckten $p$-$V$-Diagramm sind das Koexistenzgebiet von
       gasf\"ormiger und fl\"ussiger Phase zu markieren und der kritische Punkt
       einzuzeichnen. Geben Sie die kritischen Gr\"o\ss en $T_c$, $p_c$ und $V_c$ an.
 \item Tragen Sie in einem zweiten Diagramm f\"ur alle Messreihen $p \cdot V$ \"uber $1/V$
   auf und extrapolieren Sie linear bis nach $1/V = 0$. Ermitteln Sie die
   Stoffmenge $n$ entsprechend Gleichung \eqref{LIM} aus den Schnittpunkten der
   Extrapolationsgeraden mit der Ordinate.

 \begin{refequation}{1}
  \lim\limits_{V \to \infty} (p \cdot V) = n \cdot R \cdot T \label{LIM}
 \end{refequation}
 Berechnen Sie nach Gleichung \eqref{KOEFF} die van-der-Waals-Koeffizienten $a$ und $b$.
 \begin{refequation}{2}
  a = 3 p_k \cdot \kla{ \frac{V_{k}}{n} }^2
  \quad\text{und}\quad
  b = \frac{V_k}{3n} \label{KOEFF}
\end{refequation}
\end{enumerate}

Zus\"atzlich f\"ur Physikstudenten:

\begin{enumerate}\setcounter{enumi}{3}
 \item Zeichnen Sie die Dampfdruckkurve laut Gleichung \eqref{DDK}. Tragen Sie
   dabei $\ln p$ \"uber $1/T$ auf und bestimmen Sie aus dem Anstieg der
   Geraden die molare Verdampfungsenthalpie.

 \begin{refequation}{3}
  p_S (T) = p_0 \cdot \eu{- \frac{L}{RT}} \label{DDK}
 \end{refequation}

  Bestimmen Sie zu jeder Isotherme mit $T<T_c$ die Verdampfungsw\"arme
  nach Gleichung  \eqref{CCG}. Die Volumina $V_g$ und $V_f$ sind dem
  p-V-Diagramm zu entnehmen. 
  Stellen Sie die ermittelten Gr\"o\ss en $T$, $p_S$, $\diff p/\diff T$,
  $V_g-V_f$ sowie $Q$ \"ubersichtlich in einer Tabelle zusammen.

 \begin{refequation}{4}
  Q = T \cdot \dquot{p_S}{T}(V_g - V_f) \label{CCG}
 \end{refequation}

\end{enumerate}

Diskutieren Sie alle Ergebnisse und vergleichen Sie ggf. mit Literaturwerten.

\end{document}

