\documentclass[platz]{tudphygp}

\versuch{Thermische Ausdehnung}{TA}

\begin{document}
\maketitle

\subsection*{Aufgabenstellung}

\begin{enumerate}
 \item Bestimmen Sie den \emph{mittlere L�ngenausdehnungskoeffizienten} $(\overline\alpha \pm \Delta\overline\alpha)$ eines Metallstabes im Temperaturbereich zwischen Raumtemperatur und \SI{60}{\degC} mit Hilfe des \emph{Hebelverfahrens}.
 \item Bestimmen Sie den \emph{mittleren Volumenausdehnungskoffizienten} $(\overline\gamma \pm\Delta\overline\gamma)$ einer Fl�ssigkeit im Temperaturbereich zwischen Raumtemperatur und \SI{60}{\degC} mit Hilfe des \emph{Dilatometers}.
\end{enumerate}

\subsection*{L�ngenausdehungskoeffizient}

\begin{enumerate}
 \item Messen Sie die Ausgangsl�nge $l_0$ des Stabes bei Raumtemperatur $T_0$.
 \item Bringen Sie den Stab in der Versuchsapparatur in ein Wasserbad, dessen Temperatur mittels Themostat veriiert werden kann. Ein Stabende \emph{liegt} an einem festen Anschlag \emph{an}. Die thermische L�ngen�nderung des Stabes wird von seinem anderen Ende �ber einen Hebel (Hebelverh�ltnis $1:5$) auf eine Messuhr (Genauigkeitsklasse II; Messbereich $\num0\ldots\SI{10}{mm}$) �bertragen.
 \item W�hlen Sie eine \emph{geeignete} Bezugstemperatur $T_1$, bei der Sie die Nullstellung der Messuhr festlegen.
 \item Messen Sie die L�ngen�nderung des Metallstabes bis \SI{60}{\degC} in \emph{sechs} Schritten.
 \item Stellen Sie die \emph{relativen} L�ngen�nderungen $\Delta l / l_0$ �ber der Temperaturdifferenz $\Delta T$ dar.
  \[
   \frac{\Delta l}{l_0} = \frac{l(T) - l(T_1)}{l_0} \quad\text{und}\quad \Delta T = T - T_1
  \]
 \item Ermitteln Sie aus dem Zusammenhang $\Delta l / l_0 = f(\Delta T)$ mittels Geradenausgleich den mittleren Anstieg der Messkurve und damit den mittleren L�ngenausdehnungskoeffizienten:
  \[
   \overline\alpha = \frac{l(T_2) - l(T_1)}{l(T_1)} \cdot \frac1{T_2 - T_1}
  \]
  Hierbei gilt in guter N�herung $l(T_1) \approx l(T_0) = l_0$, falls $T_1$ geeignet gew�hlt wurde (warum?).
 \item Ermitteln Sie aus dem minimalen und dem maximalen Anstieg den Fehler $\Delta\overline\alpha$.
\end{enumerate}

\subsection*{Volumenausdehnungskoffizient}

\begin{table}[b] \centering
 \begin{tabular}{llcc}
  Nr. & Fl�ssigkeit      & $\varrho_\mathrm{Fl}$ / \SI{}{g.cm^{-3}} & $m_\mathrm{K}$ / \SI{}g \\\hline
  1   & Ameisens�ure     & \num{1,238}                          & \num{41,030} \\
  2   & Paraffin�l       & \num{0,826}                          & \num{45,279} \\
  3   & L�vakol          & \num{0,869}                          & \num{42,210} \\
  4   & Iso-(2-Propanol) & \num{0,781}                          & \num{40,720}
 \end{tabular}
 \caption{Dichte $\varrho_\mathrm{Fl}$ von Fl�ssigkeiten bei \SI{25}{\degC}; Kolbenleermasse $m_\mathrm{K}$}\label{tab::dichte}
\end{table}

\begin{enumerate}
 \item Bestimmen Sie die Gesamtmasse $m$ der Messanordnung als Summe der Masse des leeren Kolbens $m_\mathrm{K}$ und der Masse der zu untersuchenden Fl�ssigkeit $m_\mathrm{Fl}$. Erhalten Sie daraus das Volumen $V_0$ der Fl�ssigkeit. Die dazu n�tigen Werte entnehmen Sie Tabelle \ref{tab::dichte}.
 \item Platzieren Sie den Glaskolben etwa \SI{20}{mm} �ber den Boden des Gef��es mit dem Wasserbad, dessen Temperatur �ber ein Thermostat geregelt werden kann.
 \item W�hlen Sie ein \emph{geeignete} Bezugstemperatur $T_1$.
 \item Nehmen Sie \emph{sechs} Messwerte der relativen Volumenausdehnung $\Delta V / V_0$ bei der Erw�rmung bis \SI{60}{\degC} auf, indem Sie die �nderung direkt an der Skala am Steigrohr ablesen. Nutzen Sie w�hrend der Erw�rmung der Thermostatfl�ssigkeit den Magnetr�hrer im Glaskolben.
  \[
   \frac{\Delta V}{V_0} = \frac{V(T) - V(T_1)}{V_0}
  \]
  Wurde $T_1$ geeignet gew�hlt, gilt wiederum in guter N�herung $V(T_1) \approx V_0$ (warum?).
 \item Stellen Sie die relative Volumen�nderung $\Delta V / V_0$ �ber der Temperatur�nderung $\Delta T$ grafisch dar.
 \item Ermitteln Sie aus dem Zusammenhang $\Delta V / V_0 = f(\Delta T)$ mittels Geradenausgleich den mittlere Anstieg der Messkurve. Nutzen f�r den Geradenausgleich einmal die \emph{grafische Methode} und einmal die \emph{Methode der kleinsten Fehlerquadratsumme}. Berechnen Sie aus dem Anstieg den mittleren Volumenausdehnungskoffizienten $\overline\gamma$:
  \[
   \overline\gamma = \frac{V(T_2) - V(T_1)}{V(T_1)} \cdot \frac1{T_2 - T_1} + 3\alpha_\mathrm{Glas} \cdot \frac{V(T_2)}{V(T_1)} \quad\text{mit}\quad \alpha_\mathrm{Glas} = \SI{8e-6}{K^{-1}}
  \]
 \item Bestimmen Sie den Fehler $\Delta\overline\gamma$ mittels grafischer Methode.
\end{enumerate}

%TODO: Foto des Versuches

\end{document}
